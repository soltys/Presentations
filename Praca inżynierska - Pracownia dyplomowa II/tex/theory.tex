\begin{frame}
\frametitle{Zawartość części teoretycznej}
Opis użytych narzędzi podczas budowania aplikacji biznesowej razem z uzasadnieniem.
Między innymi:
\begin{itemize}
\item Zadania aplikacji biznesowej
\item Windows 8 i Platforma Windows Runtime
\item Visual Studio
\item Języki programowania
\item Języki opisu interfejsu użytkownika
\item Baza danych SQLite
\end{itemize}
\end{frame}


\begin{frame}
\frametitle{Zadania aplikacji biznesowej}
Przykładowe zadania aplikacji biznesowej
\begin{itemize}
\item	Bezpieczny dostęp do danych w firmie
\item	Przeglądanie danych pochodzących z firmy
\item	Umieszczanie danych
\end{itemize}
\end{frame}

\begin{frame}
\frametitle{Windows 8}
Windows 8 jest wersją systemu operacyjnego Microsoft Windows, produkowanego przez Microsoft przeznaczoną do użytku na komputerach osobistych, włączając w to domowe i firmowe komputery stacjonarne, laptopy i tablety PC.
\end{frame}

\begin{frame}
\frametitle{Windows Runtime}
Windows Runtime lub w skrócie WinRT jest jednolitą platformą do tworzenia aplikacji w systemie operacyjnym Windows 8. Windows Runtime pozwala na natywne programowanie w języku C++/CX oraz na programowanie przy użyciu języków zarządzalnych takich jak \Csharp i \texttt{VB.NET}, a także \texttt{JavaScript}. Dzięki platformie Windows Runtime programy mogą być natywnie uruchamiane na procesorach x86 oraz ARM w trybie piaskownicy.
\end{frame}

\begin{frame}
\frametitle{Windows Runtime -- biblioteka}
Windows Runtime jest oparte na COM (Component Object Model). Wykorzystanie COM pozwala na utworzenie wspólnego interfejsu programistycznego z wszystkimi językami programowania.
\end{frame}

\begin{frame}
\frametitle{Visual Studio}
\begin{itemize}
\item	Wymagana licencja deweloperska
\item	Przykładowy kod aplikacji
\item	Wizualny edytor interfejsu użytkownika
\item	Edytor kodu
\item	Dystrybucja -- deployment
\item	Lokalizacja
\item 	Debugowanie i testowanie (testy jednostkowe)
\end{itemize}
\end{frame}

\begin{frame}
\frametitle{Języki programowania}
\begin{itemize}
\item	JavaScript
\item	\Csharp
\item	\texttt{C++/CX}

\end{itemize}
\end{frame}

\lstset{
  basicstyle=\ttfamily,
  columns=fullflexible,
  showstringspaces=false,
  commentstyle=\color{gray}\upshape
}

\begin{frame}[fragile]
\frametitle{JavaScript}
\begin{adjustbox}{width=\textwidth}
\begin{lstlisting}

function onDataChanged(e) { 
        var reading = e.reading; 
 
        document.getElementById("eventOutputX").innerHTML = reading.accelerationX.toFixed(2); 
        document.getElementById("eventOutputY").innerHTML = reading.accelerationY.toFixed(2); 
        document.getElementById("eventOutputZ").innerHTML = reading.accelerationZ.toFixed(2); 
    }  
\end{lstlisting}
\end{adjustbox}

\end{frame}

\begin{frame}[fragile]
\frametitle{\Csharp}
\begin{adjustbox}{width=\textwidth}
\begin{lstlisting}
async private void ReadingChanged(object sender, AccelerometerReadingChangedEventArgs e) 
        { 
            await Dispatcher.RunAsync(CoreDispatcherPriority.Normal, () => 
            { 
                AccelerometerReading reading = e.Reading; 
                ScenarioOutput_X.Text = String.Format("{0,5:0.00}", reading.AccelerationX); 
                ScenarioOutput_Y.Text = String.Format("{0,5:0.00}", reading.AccelerationY); 
                ScenarioOutput_Z.Text = String.Format("{0,5:0.00}", reading.AccelerationZ); 
            }); 
        } 
\end{lstlisting}
\end{adjustbox}
\end{frame}


\begin{frame}[fragile]
\frametitle{\texttt{C++/CX}}
\begin{adjustbox}{width=\textwidth}
\begin{lstlisting}
void Scenario1::ReadingChanged(Accelerometer^ sender, AccelerometerReadingChangedEventArgs^ e) 
{ 
    // We need to dispatch to the UI thread to display the output 
    Dispatcher->RunAsync( 
        CoreDispatcherPriority::Normal, 
        ref new DispatchedHandler( 
            [this, e]() 
            { 
                AccelerometerReading^ reading = e->Reading; 
 
                ScenarioOutput_X->Text = reading->AccelerationX.ToString(); 
                ScenarioOutput_Y->Text = reading->AccelerationY.ToString(); 
                ScenarioOutput_Z->Text = reading->AccelerationZ.ToString(); 
            }, 
            CallbackContext::Any 
            ) 
        ); 
} 
\end{lstlisting}
\end{adjustbox}

\end{frame}

\begin{frame}
\frametitle{Języki opisu interfejsu użytkownika}
\begin{itemize}
\item	XAML
\item	HTML
\end{itemize}
\end{frame}



\lstdefinelanguage{XML}
{
  morestring=[b]",
  morestring=[s]{>}{<},
  morecomment=[s]{<?}{?>},
  stringstyle=\color{black},
  identifierstyle=\color{darkblue},
  keywordstyle=\color{cyan},
  morekeywords={xmlns,version,type}% list your attributes here
}

\begin{frame}[fragile]
\frametitle{XAML}
 \begin{adjustbox}{width=\textwidth}
\begin{lstlisting}[language=XML,basicstyle=\ttfamily]

<Grid x:Name="LayoutRoot" Background="White" 
      HorizontalAlignment="Left" VerticalAlignment="Top"> 
        <Grid.RowDefinitions> 
            <RowDefinition Height="Auto"/> 
            <RowDefinition Height="*"/> 
        </Grid.RowDefinitions> 
        <Grid x:Name="Input" Grid.Row="0"> 
            <Grid.RowDefinitions> 
                <RowDefinition Height="Auto"/> 
                <RowDefinition Height="*"/> 
            </Grid.RowDefinitions> 
            <TextBlock x:Name="Scenario1Input"  TextWrapping="Wrap" 
            Grid.Row="0" Style="{StaticResource BasicTextStyle}" HorizontalAlignment="Left" > 
                Displays the profile information for the Internet Connection Profile. 
            </TextBlock> 
            <StackPanel Orientation="Horizontal" Margin="0,10,0,0" Grid.Row="1"> 
                <Button x:Name="InternetConnectionProfileButton" 
                Content="Get Internet Connection Profile Info" 
                Margin="0,0,10,0" Click="InternetConnectionProfile_Click"/> 
            </StackPanel> 
        </Grid> 
</Grid> 
\end{lstlisting}
\end{adjustbox}
\end{frame}

\begin{frame}[fragile]
\frametitle{HTML}
 \begin{adjustbox}{width=\textwidth}
\begin{lstlisting}[language=XML,basicstyle=\ttfamily]
<html> 
<head> 
    <meta charset="utf-8" /> 
    <title>SDK Sample</title> 
    <!-- WinJS references --> 
    <link rel="stylesheet" href="//Microsoft.WinJS.1.0/css/ui-light.css" /> 
    <script src="//Microsoft.WinJS.1.0/js/base.js"></script> 
    <script src="//Microsoft.WinJS.1.0/js/ui.js"></script> 
 
    <!-- SDK sample framework references --> 
    <link rel="stylesheet" href="/sample-utils/sample-utils.css" /> 
    <link rel="stylesheet" href="/css/default.css" /> 
    <script src="/sample-utils/sample-utils.js"></script> 
    <script src="/js/default.js"></script> 
</head> 
<body role="application"> 
    <div id="rootGrid"> 
        <div id="header" role="contentinfo" data-win-control="WinJS.UI.HtmlControl" 
        data-win-options="{uri: '/sample-utils/header.html'}"></div> 
        <div id="content"> 
            <h1 id="featureLabel"></h1> 
            <div id="contentHost"></div> 
        </div> 
        <div id="footer" data-win-control="WinJS.UI.HtmlControl" 
        data-win-options="{uri: '/sample-utils/footer.html'}"></div> 
    </div> 
</body> 
</html> 
\end{lstlisting}
\end{adjustbox}
\end{frame}

\begin{frame}
\frametitle{SQLite}
SQLite – to system zarządzania bazą danych oraz biblioteka C implementująca taki system, obsługująca język SQL (ang. Structured Query Language). Została stworzona przez Richarda Hippa i jest dostępna na licencji public domain.\\
~\\
https://github.com/praeclarum/sqlite-net
\end{frame}
