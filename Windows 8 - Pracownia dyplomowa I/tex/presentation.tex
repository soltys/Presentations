\documentclass{beamer}

\usepackage{polski}
\usepackage[utf8]{inputenc}
\usepackage{graphicx}
\usepackage{listings}
\usepackage{verbatim} 
\usepackage{color}
\definecolor{orange}{HTML}{FF7F00}
\lstset{
language=C,
basicstyle=\ttfamily\scriptsize,
columns=fullflexible,
keepspaces,
breaklines,
tabsize=3, 
showstringspaces=false,
extendedchars=true,
keywordstyle=\color{blue}\ttfamily,
stringstyle=\color{red}\ttfamily,
commentstyle=\color{orange}\ttfamily}

\usepackage{relsize}
\usepackage{lipsum}

%c from texinfo.tex
\def\ifmonospace{\ifdim\fontdimen3\font=0pt }

\newcommand{\CC}{C\nolinebreak\hspace{-.05em}\raisebox{.4ex}{\tiny\bf +}\nolinebreak\hspace{-.10em}\raisebox{.4ex}{\tiny\bf +}}
\def\C++{{C\nolinebreak[4]\hspace{-.05em}\raisebox{.4ex}{\tiny\bf ++}}}

%c C sharp
\def\Csharp{%
\ifmonospace%
    C\#%
\else%
    C\kern-.1667em\raise.30ex\hbox{\smaller{\#}}%
\fi%
\spacefactor1000 }

\title{Programowanie w Windows 8}   
\subtitle{Pracownia dyplomowa I}
\author{Paweł Sołtysiak} 
\institute{Wydział Informatyki\\ Zachodniopomorski Uniwersytet Technologiczny w Szczecinie}
\date{\today} 
\usetheme{PaloAlto}

\begin{document}


\begin{frame}
\titlepage
\end{frame} 


\begin{frame}
\frametitle{Agenda} 
\tableofcontents
\end{frame} 

\section{Microsoft Windows 8}
\begin{frame}
\frametitle{Windows 8} 
Windows 8 to nowy system od Microsoft$\textsuperscript{\textregistered}$, którego celem było wprowadzenie jednej platformy dla wszystkich urządzeń od komputerów do tabletów.
\end{frame}

\section{Narzędzia do programowania}
\begin{frame}
\frametitle{Visual Studio 2012} 
Tylko przy użyciu Visual Studio 2012 (i Windows 8), można tworzyć aplikacje Windows Store Apps.
\end{frame}

\subsection{Języki programowania}
\begin{frame}
\frametitle{Dostępne języki programowania} 
\begin{itemize}
\item JavaScript oraz HTML (można użyć wszystkich istniejących bibliotek JavaScript!)
\item C\# oraz XAML
\item C++/CX oraz XAML (nie mylić z C++/CLI!)
\end{itemize} 
\end{frame}

\begin{frame}
\frametitle{Każdy język posiada inny cel} 
\begin{itemize}
\item JavaScript oraz HTML - Szybkie tworzenie aplikacji, narzędzie dla Web Developerów.
\item C\# oraz XAML
\item C++/CX oraz XAML - Dla aplikacji w których ważna jest wydajność.
\end{itemize} 
\end{frame}

\subsection{Platforma Windows Runtime}
\begin{frame}[fragile]
\frametitle{Platforma Windows Runtime}
Windows Runtime bazuje na COM (Component Object Model)
\\
Platforma Windows Runtime dopasowuje się do języka programowania!
\end{frame}

\begin{frame}[fragile]
\frametitle{\Csharp}
 \begin{lstlisting}
StorageFolder picturesFolder = KnownFolders.PicturesLibrary;
StringBuilder outputText = new StringBuilder();

IReadOnlyList<IStorageItem> itemsList = 
    await picturesFolder.GetItemsAsync();

foreach (var item in itemsList)
{
    if (item is StorageFolder)
    {
        outputText.Append(item.Name + " folder\n");
    }
    else
    {
        outputText.Append(item.Name + "\n");
    }
}
    \end{lstlisting}  
\end{frame}

\begin{frame}[fragile]
\frametitle{\C++}
 \begin{lstlisting}
StorageFolder^ picturesFolder = KnownFolders::PicturesLibrary;
auto outputString = make_shared<wstring>();

create_task(picturesFolder->GetItemsAsync())        
    .then ([this, outputString] (IVectorView<IStorageItem^>^ items)
{        
    for ( unsigned int i = 0 ; i < items->Size; i++)
    {
        *outputString += items->GetAt(i)->Name->Data();
        if(items->GetAt(i)->IsOfType(StorageItemTypes::Folder))
        {
            *outputString += L"  folder\n";
        }
        else
        {
            *outputString += L"\n";
        }
        m_OutputTextBlock->Text = ref new String((*outputString).c_str());
    }
});
    \end{lstlisting}  
\end{frame}

\begin{frame}[fragile]
\frametitle{JavaScript}
\begin{lstlisting}
// Get folder
var picturesLibrary = Windows.Storage.KnownFolders.picturesLibrary;
//  Get folder contents
picturesLibrary.getItemsAsync().then(function (items) {
    // Then perform tasks with the list of files and folders that was retrieved
    // For example, display name of containing folder and count of items in folder
    outputHeader(picturesLibrary.name, items.size);

    // For example, display info for each item
    items.forEach(function (item) { 
        if (item.isOfType(Windows.Storage.StorageItemTypes.folder)) { 
            output(id(picturesLibrary.name), item.name + "\\"); 
        } 
        else { 
            output(id(picturesLibrary.name), item.fileName); 
        } 
    }); 
});
\end{lstlisting}
\end{frame}

\section{Demo}
\subsection{XAML}
\begin{frame}
\frametitle{Czym jest XAML?}
XAML -- to język opisu interfejsu użytkownika\\ opartym na \ldots XML
\end{frame}


\subsection{Model View ViewModel}
\begin{frame}
\frametitle{Czym jest Model View ViewModel?}
Wzorzec architektoniczny -- Służy do rozdzielenia warstwy widoku (GUI) od logiki (back-end). \\ 
\end{frame}

\begin{frame}
\frametitle{Delivery App -- demo}

\begin{Huge}
Delivery App - DEMO
\end{Huge}
\\~\\
\begin{large}
Przy użyciu \Csharp oraz XAML
\end{large}

\end{frame}


\section{Zakończenie}
\begin{frame}
\frametitle{Dziękuje za uwagę...}
Prezentacja (w \LaTeX) i demo są dostępne online \href{http://github.com/soltys}{\beamergotobutton{github.com/soltys}}. \\
Bezpośredni link: \href{http://snipurl.com/windows8-prezentacja}{http://snipurl.com/windows8-prezentacja}\\
\\
Pytania?

\end{frame}

\end{document}
