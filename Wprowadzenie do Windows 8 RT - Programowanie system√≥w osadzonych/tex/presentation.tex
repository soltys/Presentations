\documentclass{beamer}

\usepackage{polski}
\usepackage[utf8]{inputenc}
\usepackage{graphicx}
\usepackage{listings}
\usepackage{verbatim} 
\usepackage{color}
\definecolor{orange}{HTML}{FF7F00}
\lstset{
language=C,
basicstyle=\ttfamily\scriptsize,
columns=fullflexible,
keepspaces,
breaklines,
tabsize=3, 
showstringspaces=false,
extendedchars=true,
keywordstyle=\color{blue}\ttfamily,
stringstyle=\color{red}\ttfamily,
commentstyle=\color{orange}\ttfamily}

\usepackage{relsize}
\usepackage{lipsum}

%c from texinfo.tex
\def\ifmonospace{\ifdim\fontdimen3\font=0pt }

\newcommand{\CC}{C\nolinebreak\hspace{-.05em}\raisebox{.4ex}{\tiny\bf +}\nolinebreak\hspace{-.10em}\raisebox{.4ex}{\tiny\bf +}}
\def\C++{{C\nolinebreak[4]\hspace{-.05em}\raisebox{.4ex}{\tiny\bf ++}}}

%c C sharp
\def\Csharp{%
\ifmonospace%
    C\#%
\else%
    C\kern-.1667em\raise.30ex\hbox{\smaller{\#}}%
\fi%
\spacefactor1000 }

\title{Wprowadzenie do programowania w WinRT (Windows 8)}   
\subtitle{Programowanie systemów osadzonych}
\author[Paweł Sołtysiak]
       {Paweł Sołtysiak \\ \texttt{psoltysiak@uznam.org}}
%\author{Paweł Sołtysiak\\ \texttt{psoltysiak@uznam.org}} 
\institute{Wydział Informatyki\\ Zachodniopomorski Uniwersytet Technologiczny w Szczecinie}
\date{28 października 2013} 
\usetheme{PaloAlto}

\begin{document}


\begin{frame}
\titlepage
\end{frame} 


\begin{frame}
\frametitle{Agenda} 
\tableofcontents
\end{frame} 

\section{Microsoft Windows 8}
\begin{frame}
\frametitle{Windows 8} 
Windows Runtime to nowy podsystem od Microsoft$\textsuperscript{\textregistered}$, którego celem było wprowadzenie jednej platformy dla wszystkich urządzeń od komputerów do tabletów.
\end{frame}

\section{Narzędzia do programowania}
\begin{frame}
\frametitle{Visual Studio 2012} 
Tylko przy użyciu Visual Studio 2012 i starszych, można tworzyć aplikacje Windows Store Apps.
\end{frame}

\subsection{Języki programowania}
\begin{frame}
\frametitle{Dostępne języki programowania} 
\begin{itemize}
\item JavaScript oraz HTML
\item C\# oraz XAML
\item C++/CX oraz XAML (nie mylić z C++/CLI!)
\end{itemize} 
\end{frame}

\begin{frame}
\frametitle{Każdy język posiada inny cel} 
\begin{itemize}
\item JavaScript oraz HTML - Szybkie tworzenie aplikacji, narzędzie dla Web Developerów.
\item C\# oraz XAML
\item C++/CX oraz XAML - Dla aplikacji w których ważna jest wydajność.
\end{itemize} 
\end{frame}

\subsection{Platforma Windows Runtime}
\begin{frame}[fragile]
\frametitle{Platforma Windows Runtime}
Windows Runtime bazuje na COM (Component Object Model)
\\
Platforma Windows Runtime dopasowuje się do języka programowania!
\end{frame}


\section{Demo}
\subsection{Hello World}
\begin{frame}
\frametitle{Hello World -- C\#}
Demo będzie zawierać:
\begin{itemize}
\item Wyświetlenie tekstu
\item Przycisk z akcję
\item Trochę o XAML
\end{itemize}
\end{frame}


\begin{frame}
\frametitle{Hello World -- demo}

\begin{Huge}
Hello World -- DEMO
\end{Huge}
\\~\\
\begin{large}
Przy użyciu C\# oraz XAML
\end{large}

\end{frame}


\section{Zakończenie}
\begin{frame}
\frametitle{Dziękuje za uwagę!}
Prezentacja (w \LaTeX) jest dostępna online \href{http://github.com/soltys}{\beamergotobutton{github.com/soltys}}. 
~\\
~\\
Pytania?

\end{frame}

\end{document}
