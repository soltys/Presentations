\documentclass{beamer}

\usepackage{polski}
\usepackage[utf8]{inputenc}
\usepackage{graphicx}

\title{Programowanie w Windows 8}   
\subtitle{Pracownia dyplomowa I}
\author{Paweł Sołtysiak} 
\institute{Wydział Informatyki\\ Zachodniopomorski Uniwersytet Technologiczny w Szczecinie}
\date{\today} 
\usetheme{PaloAlto}



\begin{document}
\begin{frame}
\titlepage
\end{frame} 


\begin{frame}
\frametitle{Agenda} 
\tableofcontents
\end{frame} 

\section{Microsoft Windows 8}
\begin{frame}
\frametitle{Windows 8} 
Windows 8 to nowy system od Microsoft$\textsuperscript{\textregistered}$, którego celem było wprowadzenie jednej platformy dla wszystkich urządzeń od komputerów do tabletów.
\end{frame}

\section{Narzędzia do programowania}
\begin{frame}
\frametitle{Visual Studio 2012} 
Tylko przy użyciu Visual Studio 2012 (i Windows 8), można tworzyć aplikacje Windows Store Apps.
\end{frame}

\subsection{Języki programowania}
\begin{frame}
\frametitle{Dostępne języki programowania} 
\begin{itemize}
\item JavaScript oraz HTML (można użyć wszystkich istniejących bibliotek JavaScript!)
\item C\# oraz XAML
\item C++/CX oraz XAML (nie mylić z C++/CLI!)
\end{itemize} 
\end{frame}

\begin{frame}
\frametitle{Każdy język posiada inny cel} 
\begin{itemize}
\item JavaScript oraz HTML - Szybkie tworzenie aplikacji, narzędzie dla Web Developerów.
\item C\# oraz XAML - Wykorzystanie już znanych umiejętności.
\item C++/CX oraz XAML - Dla aplikacji w których ważna jest wydajność.
\end{itemize} 
\end{frame}

\subsection{Biblioteka Windows Runtime}
\begin{frame}
\frametitle{Biblioteka Windows Runtime}
Biblioteka na którą trzeba patrzyć z perspektywy języka programowania. API dopasowuje się do języka programowania!
\end{frame}

\section{Dema}
\subsection{XAML}
\begin{frame}
\frametitle{Czym jest XAML?}
XAML - to język opisu interfejsu użytkownika\\ opartym na \ldots XML
\end{frame}

\begin{frame}
\frametitle{XAML - demo}
\begin{Huge}
DEMO
\end{Huge}
\end{frame}

\subsection{Model View ViewModel}
\begin{frame}
\frametitle{Czym jest Model View ViewModel?}
Wzorzec architektoniczny - Służy do rozdzielenia warstwy widoku (GUI) od logiki (back-end).\pause \\ 
\\
Informacje dla osób programujących w WPF lub Silverlight: Data Binding jest z Silverlight 2 \pause \\
\textbf{:-(}
\end{frame}

\begin{frame}
\frametitle{MVVM - demo}
\begin{Huge}
DEMO
\end{Huge}
\end{frame}
\section{Zakończenie}
\begin{frame}
\frametitle{Dziękuje za uwagę...}
Prezentacja i demo są dostępne online \href{http://github.com/soltys}{\beamergotobutton{github.com/soltys}}. \\
Bezpośredni link: \href{http://snipurl.com/windows8-prezentacja}{http://snipurl.com/windows8-prezentacja}\\
\\
Pytania?
\end{frame}

\end{document}
